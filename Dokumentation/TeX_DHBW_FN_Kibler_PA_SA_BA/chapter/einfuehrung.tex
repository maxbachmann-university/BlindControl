\chapter{Einführung}
\label{cha:Einführung}
Das Projekt BlindControl entstand im Rahmen der Mikrocontroller Vorlesung des Dozenten Hans Jürgen Herpel an der DHBW Ravensburg Campus Friedrichshafen. Ziel ist es, eine Projektentwicklung anhand eines selbstgewählten Beispiels zu durchlaufen, um die entsprechenden Vorgänge zu erlernen und zu dokumentieren. Das fertige Projekt kann in dem zugehörigen \href{https://github.com/maxbachmann/BlindControl}{Git Repository} auf Github angesehen und verwendet werden. Zudem gibt es zu dem Projekt ein ausführliches \href{https://github.com/maxbachmann/BlindControl/wiki}{Github Wiki}.

% Was ist das Problem?
\section{Heranführung an die Problemstellung}
Die dem Projekt zugrundeliegende Idee ist die automatische Ansteuerung einer Jalousie mittels Home Assistant. Viele Jalousien sind nur manuell zu steuern oder nur mit teurer Technik automatisiert. Dennoch birgt eine automatische Steuerung einen hohen Comfort, auf den auch andere Anwender nicht verzichten sollten. Anwendungsbeispiele sind Folgende:
\begin{itemize}
	\item Bei zu hoher Sonneneinstrahlung verdunkelt die Jalousie automatisch, um eine angenehme Raumtemperatur auch im Sommer zu wahren.
	\item Jalousien sind durch ihren Aufbau sehr windanfällig und können leicht Schaden nehmen, oder gar anwesende Personen gefährden, sollte ein Sturm aufziehen. Auch hier kann eine Steuerung reagieren und die Jalousien automatisiert hochfahren und so zu schützen.
	\item Zusätzlich zu der automatisierten Steuerung kann es auch vorkommen, dass ein Nutzer beispielsweise auf dem Sofa liegt und durch die Sonnen geblendet wird. Um nicht aufstehen und zum Schalter an der Wand laufen zu müssen, könnte eine Heimautomatisierung zum Einsatz kommen, in der eine manuelle Steuerung der Jalousien integriert ist. Damit könnten diese mit dem Smartphone, das viele Benutzer immer bei sich haben, gesteuert werden.
	\item Auch die nachträgliche Installation solcher Technik kann Probleme bereiten, wenn beispielsweise Leitungen von der Jalousie zu dem Steuerungsserver gelegt werden müssen. Dagegen ist eine auf Funk basierte Kommunikation der einzelnen Komponenten optimal.
\end{itemize}
Auf Grundlage der genannten Anwendungssituationen wurden für das Projekt BlindControl die Anforderungen(Requirements) entworfen, sowie die Hardware und Software ausgewählt.

% Wie sieht der Lösungsraum aus?
\section{Lösungsansätze}
Um alle genannten Funktionen und Eigenschaften zusammen nutzen zu können, wird eine zentrale Steuereinheit benötigt, die alle Jalousien verwaltet. Auf diese sollte der Benutzer zugreifen können, um die manuelle Steuerung zu übernehmen. Zudem sollten hier alle Befehle an die Jalousien zentral gesendet werden, die damit eine eigene Ansteuerung brauchen. Um auf Umwelteinflüsse, wie Wind und Helligkeitswert zu reagieren. Dafür wird ein Sensor-Modul gebraucht, um die Steuereinheit auch dezentral platzieren zu können. Das gesamte System sollte möglichst einfach installierbar und anpassbar sein, sowie durch Funkbetrieb Kabelleitungen zwischen den Untersystemen vermeiden und eine einfache Einrichtung gewährleisten.

% Welche Lösung wurde gewählt?
Um ein eben beschriebenes System zu gewährleisten, wurde das System in drei Untersysteme gegliedert. Die Zentralsteuerung wurde mit einem Raspberry Pi 3 realisiert, der mit Raspbian Stretch aufgesetzt wurde und mit HomeAssistant als Smarthome Verwaltung konfiguriert wurde. Um eine schnelle Software Installation zu garantieren, wurde ein Ansible Skript eingesetzt. Zusätzlich wurde Unbound für eine privatere DNS Abfrage installiert. Die Sensorwerte wurden über einen Nordic Mikrocontroller eingelesen und über ein Thread Netzwerk an den Border Router mittels MQTT übermittelt. Dieser konnte dann auf Grund dieser Daten mittels MQTT einen ESP32 Befehle zum Jalousien Stand senden.