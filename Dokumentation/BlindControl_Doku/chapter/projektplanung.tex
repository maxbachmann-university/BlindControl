\chapter{Planung des Projekts}
\label{cha:Planung}

\section{Arbeitspakete}

Es bietet sich an das Projekt in mehrere Arbeitspakete aufzuteilen. Hierzu bietet es sich an sich an den verwendeten Codemodulen zu orientieren. Dies sind in diesem Projekt das Regler Module, Das Actuator Module, das Sensor Module und Codeteile wie der Wifi-Task die für mehrere Code Module in gleicher Form benötigt werden. Innerhalb der größeren Module Actuator Module und Sensor Module bietet es sich an eine weiter Unterteilung entsprechend der Tasks vorzunehmen. Die Grafik \ref{fig:arbeitspakete} zeigt die genaue Aufteilung der Arbeitspakete.

\begin{figure}[hbt]
	\centering
	\includegraphics[width=1\linewidth]{images/arbeitspakete}
	\caption[Arbeitspakete]{Arbeitspakete}
	\label{fig:arbeitspakete}
\end{figure}

\section{Zeitplan}
Das Grafik \ref{fig:time_plan} zeigt den Zeitbedarf für die einzelnen Arbeitspakete in Form eines Gant Diagramms. Der Zeitbedarf für die einzelnen Arbeitspakete kann ebenfalls aus der Grafik \ref{fig:arbeitspakete} abgelesen werden.

\begin{figure}[hbt]
	\centering
	\includegraphics[width=1\linewidth]{images/Gant}
	\caption[Zeitplanung als Gant Diagramm]{Zeitplanung als Gant Diagramm}
	\label{fig:time_plan}
\end{figure}


\section{Kostenschätzung}
\label{cha:Planung_cost}
In der Planungsphase wurde angenommen, dass mit anderen Mikrocontrollern und damit einem höheren Aufwand gearbeitet wird. Damit ergeben sich gesamt 2200 SLOC. Die Vorlesung Mikrocontroller umfasst 64h und somit ergibt sich im Schnitt 34 lines/h, die geschrieben werden müssen. Die Kostenschätzung nach COCOMOII\_2000.4 ist in Abbildung~\ref{fig:cost_plan} zu sehen.

\begin{figure}[hbt]
	\centering
	\includegraphics[width=1\linewidth]{images/cost_plan_img}
	\caption[Kostenschätzung Planung]{Kostenschätzung bei der Planung.}
	\label{fig:cost_plan}
\end{figure}