\chapter{Planung des Projekts}
\label{cha:Planung}

\section{Arbeitspakete}

Es macht immer Sinn Projekte in mehrere Arbeitspakete aufzuteilen. Hierbei bietet es sich in diesem Projekt an sich an den verwendeten Codemodulen zu orientieren. Dies sind in diesem Projekt das Regler Module, das Actuator Module, das Sensor Module und Codeteile wie der Wifi-Task\nomenclature{Wifi}{Wireless Fidelity} die für mehrere Code Module in gleicher Form benötigt werden. Innerhalb der größeren Module Actuator Module und Sensor Module bietet es sich an eine weiter Unterteilung entsprechend der Tasks vorzunehmen. Die Grafik \ref{fig:arbeitspakete} zeigt die genaue Aufteilung der Arbeitspakete.

\begin{figure}[hbt]
	\centering
	\includegraphics[width=1\linewidth]{images/arbeitspakete}
	\caption[Arbeitspakete]{Arbeitspakete}
	\label{fig:arbeitspakete}
\end{figure}

\section{Zeitplan}
Das Grafik \ref{fig:time_plan} zeigt den Zeitbedarf für die einzelnen Arbeitspakete in Form eines Gant Diagramms. Der Zeitbedarf für die einzelnen Arbeitspakete kann ebenfalls aus der Grafik \ref{fig:arbeitspakete} abgelesen werden.

\begin{figure}[hbt]
	\centering
	\includegraphics[width=1\linewidth]{images/Gant}
	\caption[Zeitplanung als Gant Diagramm]{Zeitplanung als Gant Diagramm}
	\label{fig:time_plan}
\end{figure}


\section{Kostenschätzung}
\label{cha:Planung_cost}
Die Aufwandsschätzung in der Planungsphase betrug 2200 SLOC\nomenclature{SLOC}{Source Lines Of Code}. Die Vorlesung Mikrocontroller umfasst 64h, woraus sich ein Schnitt von 34 lines/h ergibt. Die Kostenschätzung nach COCOMOII\_2000.4 ist in Abbildung~\ref{fig:cost_plan} zu sehen.

\begin{figure}[hbt]
	\centering
	\includegraphics[width=1\linewidth]{images/cost_plan_img}
	\caption[Kostenschätzung Planung]{Kostenschätzung bei der Planung.}
	\label{fig:cost_plan}
\end{figure}