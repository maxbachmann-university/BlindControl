\chapter{Fazit und Ausblick}
\label{cha:Fazit und Ausblick}

Der Aktuelle Stand ist zwar grundsätzlich betriebsbereit, wobei gegebenenfalls einige Anpassungen an die individuellen Gegebenheiten nötig ist. Dies betrifft sowohl die Helligkeitsmessung, da hier aktuell noch nicht getestet ist, ob hier die Helligkeitsstärken tatsächlich sinnvoll sind, als auch die Ansteuerung der Jalousie, da hier aktuell von einem bestimmten Schrittmotortreiber ausgegangen wird, wobei sich dies von Jalousie zu Jalousie unterscheiden kann. Das Regler Module besitzt bereits die Funktion getrennt mehrere Räume anzusteuern, wobei sowohl das Sensor Module, als auch das Actuator Module hiervon aktuell noch keinen Gebrauch machen. Insbesondere die Sicherheitsvorkehrungen die leider bei vielen IoT Projekten gänzlich vernachlässigt werden, sind bei dieser Realisierung jedoch bereits sehr gut, wobei leider aktuell noch ein eigener "Update Server" nötig ist auf den die binaries für OTA Updates abgelegt werden. Durch den modularen Aufbau lässt sich ein Großteil des Projektes auch auf andere IOT Geräte anwenden (Insbesondere die MQTTS Funktionalität und die OTA Updates).

Aus den oben genannten Gründen soll sich das Projekt in Zukunft auch mehr auf die Aspekte der sicheren Kommunikation und Updatefähigkeit von IOT Geräte konzentrieren um diese in anderen Projekten nur noch als Teilkomponenten einfügen zu müssen. Konkret sind hier für die nähere Zukunft folgende Erweiterung geplant:

\begin{itemize}
	\item Da der Code sowieso öffentlich auf Github zur Verfügung gestellt wird soll das Deployment mithilfe einer Continous Integration (CI) automatisiert werden, so dass die aktuellste binary als release zur Verfügung gestellt wird, wobei diese dann auch direkt für OTA Updates genutzt werden können.
	\item Die aktuellen Tests sollen automatisiert werden, so dass diese durch eine CI beispielsweise vor Updates des aktuellen Produktionscode möglich sind
	\item Aktuell muss der Nutzer noch selbst einen MQTT Broker aufsetzten mit TLS und username/password Authentification. Weiterhin muss er selbst entsprechende Skripte wie das Regler Module in diesem Projekt starten. Dieser Prozess soll in Zukunft automatisiert werden um so auch Personen ohne das nötige technische Hintergrundwissen eine einfache Installation zu ermöglichen
	\item Die Unterstützung von mehreren Räumen soll auf den MQTT Task ausgeweitet werden um so Projekten direkt eine einfache Möglichkeit zur Ansteuerung von Geräten in verschiedenen Räumen zu geben
\end{itemize}

Weitere Erweiterungen sind in vielfältiger Art und Weise möglich, wobei die eben genannten aktuell die höchste Priorität besitzen.