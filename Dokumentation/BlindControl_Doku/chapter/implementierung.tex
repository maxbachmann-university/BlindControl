\chapter{Implementierung}
\label{cha:Implementierung}

\section{Interessante Code-Segmente}

\section{Gegenüberstellung Komplexität (SLOC)}
% 1. Schätzung, 2.Schätzung, Lösung
\subsection{1. Schätzung}
Am Anfang der Projektplanung wurde davon ausgegangen, dass das Aktor Modul mit einem ESP32 realisiert wird. Für den zentralen Regler sollte ein Raspberry Pi verwendet werden, der gleichzeitig auch den MQTT Broker beinhaltet. Zusätzlich dazu sollte auf dem Raspberry Pi ein OpenThread Border Router eingerichtet werden. Dieser wäre für die Verbindung mit einem NRF52840 Mikrocontroller von Nordic Semiconductors gebraucht worden. Dieser Mikrocontroller sendet seine Daten mittels Low Power Bluetooth, was ihn extrem stromsparend macht. Mit einer 3V Batterie ist eine hohe Laufzeit erreichbar. Mit einem NRF Dongle wäre er mit dem Raspberry verbunden worden. Leider hatten wir mit diesem Mikrocontroller keine Erfahrung und mussten somit unsere Komponenten neu überdenken. Diesen Aufbau schätzten wir mit 2200 SLOC ab.
\subsection{2. Schätzung}
\label{cha:projektplanung_2schaetzung}
Da die Lösung mit dem NRF52840 Mikrocontroller später als zeitlich nicht realisierbar angesehen wurde, musste über einen Ersatz diskutiert werden. Da für das Aktor Modul bereits ein ESP32 verwendet wurde, lag es nahe, für das Sensormodul ebenfalls einen solchen Mikrocontroller zu verwenden. Der Vorteil war, dass viele Tasks des Empfängermoduls übernommen werden konnten. Beide Module unterschieden sich nur in ihren Aufgaben Tasks: Das Aktor Modul sollte eine Jalousie ansteuern, das Sensor Modul möglichst stromsparend Sensordaten einlesen und versenden.
\subsection{Finale Lösung}
Um die Einrichtung des Systems zu erleichtern, wurde zusätzlich zu dem Aufbau der 2. Schätzung aus Kapitel~\ref{cha:projektplanung_2schaetzung} ein Router verwendet, an den der Raspberry Pi als Regler mit LAN\nomenclature{LAN}{Local Area Network} und die ESP32 Mikrocontroller des Sensor/Aktor Moduls mittels WLAN angeschlossen wurden. Somit kann das BlindControl System an einen schon vorhandenen Router des Heimnetzwerkes gekoppelt werden. Mit dieser Lösung wurden 1540 SLOC benötigt. Die Kostenschätzung mittels COCOMOII\_2000.4 ist in Abbildung~\ref{fig:cost_end} dargestellt. Zusätzlich zu den 64h der Vorlesung wurden weitere 8h investiert. Damit ergibt sich im Schnitt 21 lines/h. 

Im Gegensatz zu der Schätzung in Kapitel~\ref{cha:Planung_cost} wurden 660 SLOC weniger benötigt, sowie 13 lines/h mussten weniger geschrieben werden.

\begin{figure}[hbt]
	\centering
	\includegraphics[width=1\linewidth]{images/cost_end_img}
	\caption[Kosten Realität]{Reale Kosten am Ende des Projekts.}
	\label{fig:cost_end}
\end{figure}