% Präambel
\documentclass[11pt,			% Schriftgröße
a4paper,						% Papierformat
oneside, 						% einseitiges (oneside) oder zweiseitiges (twoside) Dokument
listof=totoc, 					% Tabellen- und Abbildungsverzeichnis ins Inhaltsverzeichnis
bibliography=totoc,				% Literaturverzeichnis ins Inhaltsverzeichnis aufnehmen
titlepage, 						% Titlepage-Umgebung statt \maketitle
headsepline, 					% horizontale Linie unter Kolumnentitel
%abstracton,					% Überschrift beim Abstract einschalten, Abstract muss dazu in {abstract}-Umgebung stehen
DIV18,							% auskommentieren, um den Seitenspiegel zu vergrößern
BCOR6mm,						% Bindekorrektur, die den Seitenspiegel um 6mm nach rechts verschiebt,
cleardoublepage=empty,			% Stil einer leeren eingefügten Seite bei Kapitelwechsel
parskip							% Absatzabstand bei Absatzwechsel einfügen
]{scrbook}			
\usepackage{ucs} 				% Dokument in utf8-Codierung schreiben und speichern
\usepackage[utf8x]{inputenc} 	% ermöglicht die direkte Eingabe von Umlauten
\usepackage[ngerman]{babel} 	% deutsche Trennungsregeln und Übersetzung der festcodierten Überschriften
\usepackage[T1]{fontenc} 		% Ausgabe aller zeichen in einer T1-Codierung (wichtig für die Ausgabe von Umlauten!)
\setlength{\parindent}{0ex} 	% bei neuem Abschnitt nicht einrücken
\linespread{1.2}\selectfont     % Zeilenabstand erhöhen - größere Werte als 1.2 nicht verwenden!!
\usepackage{scrpage2}			% SCR Headings verwenden
\setheadsepline{0.4pt}			% Kopfzeile Linien oben
\setfootsepline{0.4pt}			% Kopfzeile Linien unten
\pagestyle{scrheadings}			% SCR Headings einschalten
\usepackage{graphicx}  			% Einbinden von Grafiken erlauben
\usepackage[format=hang,		% Formatierungen von Unter- / Überschriften
font=normal,
labelfont=bf,
justification=RaggedRight,
singlelinecheck=true,
aboveskip=1mm
]{caption}


% Please add the following required packages to your document preamble:
\usepackage[table,xcdraw]{xcolor}
% If you use beamer only pass "xcolor=table" option, i.e. \documentclass[xcolor=table]{beamer}
% Note: It may be necessary to compile the document several times to get a multi-page table to line up properly
\usepackage{booktabs}
\usepackage{longtable}
\usepackage{pdfpages}



\usepackage{enumitem}			% Erlaubt Änderung der Nummerierung in der Umgebung enumerate

\usepackage{amsmath}			% Ergänzungen für Formeln
\usepackage{textcomp} 			% zum Einsatz von Eurozeichen u. a. Symbolen
\newcommand*\diff{\mathop{}\!\mathrm{d}}	% Differentialzeichen
\newcommand*\Diff[1]{\mathop{}\!\mathrm{d^#1}} % Differentialzeichen höherer Ableitung
\newcommand*\jj{\mathop{}\!\mathrm{j}}	% Komplexe Zahl j

\usepackage[					% Einstellunge Paket hyperref
hyperfootnotes=false			% im pfd-Output Fußnoten nicht verlinken
]{hyperref}

\usepackage{makeidx}			% Paket zur Erstellung eines Index
\usepackage[intoc]{nomencl} 	% zur Erstellung des Abkürzungsberzeichnisses

\usepackage[					% Einstellungen für Fußnoten
bottom,							% Ausrichtung unten
multiple,						% Trennung durch Seperator bei mehreren Fußnoten
hang,
marginal
]{footmisc}

\usepackage{calc}				% Paket zum Berechnen von Längen z.B. 0.8\linewidth

\usepackage{xcolor} 			% einfache Verwendung von Farben in nahezu allen Farbmodellen

\usepackage{listings}			% Darstellung von Quellcode mit den Umgebungen {lstlisting}, \lstinline und \lstinputlisting
\lstset{literate=				% Damit können Umlaute innerhalb Listings geschrieben werden
	{Ö}{{\"O}}1
	{Ä}{{\"A}}1
	{Ü}{{\"U}}1
	{ß}{{\ss}}1
	{ü}{{\"u}}1
	{ä}{{\"a}}1
	{ö}{{\"o}}1
}
\definecolor{mygreen}{rgb}{0,0.6,0}
\definecolor{mygray}{rgb}{0.5,0.5,0.5}
\definecolor{mymauve}{rgb}{0.58,0,0.82}
\lstset{ %
	backgroundcolor=\color{white},   % choose the background color; you must add \usepackage{color} or \usepackage{xcolor}; should come as last argument
	basicstyle=\footnotesize,        % the size of the fonts that are used for the code
	breakatwhitespace=false,         % sets if automatic breaks should only happen at whitespace
	breaklines=true,                 % sets automatic line breaking
	captionpos=t,                    % sets the caption-position to (b) bottom or (t) top
	commentstyle=\color{mygreen},    % comment style
	deletekeywords={...},            % if you want to delete keywords from the given language
	escapeinside={\%*}{*)},          % if you want to add LaTeX within your code
	escapeinside={(*@}{@*)},
	extendedchars=true,              % lets you use non-ASCII characters; for 8-bits encodings only, does not work with UTF-8
	frame=none,	                   	% "single" adds a frame around the code; "none"
	keepspaces=true,                 % keeps spaces in text, useful for keeping indentation of code (possibly needs columns=flexible)
	keywordstyle=\color{blue},       % keyword style
	language=[LaTeX]TeX,             % the language of the code
	morekeywords={*,nomenclature},   % if you want to add more keywords to the set
	numbers=left,                    % where to put the line-numbers; possible values are (none, left, right)
	numbersep=5pt,                   % how far the line-numbers are from the code
	numberstyle=\tiny\color{mygray}, % the style that is used for the line-numbers
	rulecolor=\color{black},         % if not set, the frame-color may be changed on line-breaks within not-black text (e.g. comments (green here))
	showspaces=false,                % show spaces everywhere adding particular underscores; it overrides 'showstringspaces'
	showstringspaces=false,          % underline spaces within strings only
	showtabs=false,                  % show tabs within strings adding particular underscores
	stepnumber=1,                    % the step between two line-numbers. If it's 1, each line will be numbered
	stringstyle=\color{mymauve},     % string literal style
	tabsize=2,	                   % sets default tabsize to 2 spaces
	title=\lstname                   % show the filename of files included with \lstinputlisting; also try caption instead of title
}

\makeindex						% Indexverzeichnis erstellen
\makenomenclature				% Abkürzungsverzeichnis erstellen

% -----------------------------------------------------------------------------------------------------------------
% Zum Aktualisieren des Abkürzungsverzeichnisses (Nomenklatur) bitte auf der Kommandozeile folgenden Befehl aufrufen :
% makeindex <Dateiname>.nlo -s nomencl.ist -o <Dateiname>.nls
% Oder besser: Kann in TexStudio unter Tools-Benutzer als Shortlink angelegt werden
% Konfiguration unter: Optionen-Erzeugen-Benutzerbefehle: makeindex -s nomencl.ist -t %.nlg -o %.nls %.nlo
% -----------------------------------------------------------------------------------------------------------------

% Hier die persönlichen Daten eingeben:

\newcommand{\titel}{BlindControl}
\newcommand{\untertitel}{Eine smarte Steuerung für Jalousien}
\newcommand{\arbeit}{Projektdokumentation}
\newcommand{\studiengang}{Elektrotechnik}
\newcommand{\studienrichtung}{Fahrzeugelektronik}
\newcommand{\studienschwerpunkt}{}
\newcommand{\autor}{Maximilian Bachmann, Marvin Eisenmann, Florian Vetter}
\newcommand{\matrikelnr}{123 456}
\newcommand{\kurs}{TFE17-2}
\newcommand{\firma}{Firma (Angabe entfällt ggf. bei Studienarbeit)}
\newcommand{\abgabe}{30.04.2019}
\newcommand{\betreuerdhbw}{Hans Juergen Herpel}
\newcommand{\betreuerfirma}{Gutachter der Firma (Angabe entfällt ggf. bei Studienarbeit)}
\newcommand{\jahr}{2019}			% für Angabe im Copyright-Vermerk der Titelseite

% Folgende Zeilen definieren Abkürzungen, um Befehle schneller eingeben zu können
\newcommand{\ua}{\mbox{u.\,a.\ }}
\newcommand{\zB}{\mbox{z.\,B.\ }}
\newcommand{\bs}{$\backslash$}

% Folgende Zeilen weden benötigt, um Tikz und PGF-Plot-Grafiken einzubinden
\usepackage{pgfplots}
\usepackage{pgfplotstable}
\pgfplotsset{compat=newest}
\usepgfplotslibrary{smithchart}
\usepackage{tikz}						% Tikz sollte nach Listings Pakete geladen werden.
\usetikzlibrary{arrows}

\hyphenation{Schrift-ar-ten}


% -------------------------------------------------------------------------------------------
%                     Beginn des Dokumenteninhalts
% -------------------------------------------------------------------------------------------
\begin{document}
\setcounter{secnumdepth}{3}				% Nummerierungstiefe fürs Inhaltsverzeichnis
\setcounter{tocdepth}{3}
\sffamily								% für die Titelei serifenlose Schrift verwenden

% ------------------------------ Titelei -----------------------------------------------------

\thispagestyle{plain}
\begin{titlepage}
\enlargethispage{4.0cm}
\sffamily 								% Serifenlose Grundschrift für die Titelseite einstellen

\parbox{0.5\linewidth}{
\begin{flushleft}
% Hier ggf. ein Logo der Firma
\end{flushleft}
}
\parbox{0.5\linewidth}{
\begin{flushright}
	\includegraphics[width=0.4\linewidth]{images/DHBW_d_R_FN_46mm_4c}\\[5ex]
\end{flushright}
}
				

\begin{center}

\huge{\textsc{\textbf{\titel}}}\\[1.5ex]
\Large{\textbf{\untertitel}}\\[5ex]
\LARGE{\textbf{\arbeit}}\\[2ex]
%\normalsize{für die Prüfung zum\\[1ex] Bachelor of Engineering}\\[3ex]
\Large{Studiengang \studiengang}\\[2ex]
\large{Studienrichtung \studienrichtung}\\[1ex]
%\large{ \studienschwerpunkt}\\[2ex]
\normalsize{Duale Hochschule Baden-Württemberg Ravensburg, Campus Friedrichshafen}\\[5ex]
von\\[1ex] \autor \\[20ex]


\end{center}

\begin{flushleft}

\begin{tabular}{ll}
Abgabedatum:					& \quad \abgabe \\
Bearbeitungszeitraum:		   		& \quad 07.01.2019 - 07.03.2019   \\ 
%Matrikelnummer: 			& \quad \matrikelnr \\ 
Kurs: 							& \quad \kurs \\
%Ausbildungsfirma:	 			& \quad \firma \\ 
%Betreuer der Ausbildungsfirma:  & \quad \betreuerfirma \\ 
%Gutachter der Dualen Hochschule: & \quad \betreuerdhbw \\ [5ex]

\end{tabular} 



\small
Copyrightvermerk:\\

Dieses Werk einschließlich seiner Teile ist \textbf{urheberrechtlich geschützt}. Jede Verwertung außerhalb der engen Grenzen des Urheberrechtgesetzes ist ohne Zustimmung des Autors unzulässig und strafbar. Das gilt insbesondere für Vervielfältigungen, Übersetzungen, Mikroverfilmungen sowie die Einspeicherung und Verarbeitung in elektronischen Systemen.
\end{flushleft}
\begin{flushright}
\copyright{} \jahr
\end{flushright}
\end{titlepage}

\cleardoublepage 				% erzeugt die Titelseite
\pagenumbering{roman}					% kleine, römische Seitenzahlen für Titelei
%\include{pages/erklaerung} 				% Einbinden der eidestattlichen Erklärung
%\chapter*{Kurzfassung} %*-Variante sorgt dafür, das Abstract nicht im Inhaltsverzeichnis auftaucht

Text Kurzfassung

\cleardoublepage
   			% Einbinden des Abstracts

\tableofcontents						% Erzeugen des Inhalsverzeichnisses
\cleardoublepage

% --------------------------------------------------------------------------------------------
%                    Inhalt der Bachelorarbeit
%---------------------------------------------------------------------------------------------
\pagenumbering{arabic}					% arabische Seitenzahlen für den Hauptteil

\rmfamily

\chapter{Einführung}
\label{cha:Einführung}
Das Projekt BlindControl entstand im Rahmen der Mikrocontroller Vorlesung des Dozenten Hans Jürgen Herpel an der DHBW Ravensburg Campus Friedrichshafen. Ziel ist es, eine Projektentwicklung anhand eines selbstgewählten Beispiels zu durchlaufen, um die entsprechenden Vorgänge zu erlernen und zu dokumentieren. Das fertige Projekt kann in dem zugehörigen \href{https://github.com/maxbachmann/BlindControl}{Git Repository} auf Github angesehen und verwendet werden. Zudem gibt es zu dem Projekt ein ausführliches \href{https://github.com/maxbachmann/BlindControl/wiki}{Github Wiki}.

% Was ist das Problem?
\section{Heranführung an die Problemstellung}
Die dem Projekt zugrundeliegende Idee ist die automatische Ansteuerung einer Jalousie mittels Home Assistant. Viele Jalousien sind nur manuell zu steuern oder nur mit teurer Technik automatisiert. Dennoch birgt eine automatische Steuerung einen hohen Komfort, auf den auch andere Anwender nicht verzichten sollten. Anwendungsbeispiele sind Folgende:
\begin{itemize}
	\item Bei zu hoher Sonneneinstrahlung verdunkelt die Jalousie automatisch, um eine angenehme Raumtemperatur auch im Sommer zu wahren.
	\item Jalousien sind durch ihren Aufbau sehr windanfällig und können leicht Schaden nehmen, oder gar anwesende Personen gefährden, sollte ein Sturm aufziehen. Auch hier kann eine Steuerung reagieren und die Jalousien automatisiert hochfahren und so schützen.
	\item Zusätzlich zu der automatisierten Steuerung kann es auch vorkommen, dass ein Nutzer beispielsweise auf dem Sofa liegt und durch die Sonnen geblendet wird. Um nicht aufstehen und zum Schalter an der Wand laufen zu müssen, könnte eine Heimautomatisierung zum Einsatz kommen, in der eine manuelle Steuerung der Jalousien integriert ist. Damit könnten diese mit dem Smartphone, das viele Benutzer immer bei sich haben, gesteuert werden.
	\item Auch die nachträgliche Installation solcher Technik kann Probleme bereiten, wenn beispielsweise Leitungen von der Jalousie zu dem Steuerungsserver gelegt werden müssen. Dagegen ist eine auf Funk basierte Kommunikation der einzelnen Komponenten optimal.
\end{itemize}
Auf Grundlage der genannten Anwendungssituationen wurden für das Projekt BlindControl die Anforderungen(Requirements) entworfen, sowie die Hardware und Software ausgewählt.

% Wie sieht der Lösungsraum aus?
\section{Lösungsansätze}
Um alle genannten Funktionen und Eigenschaften zusammen nutzen zu können, wird eine zentrale Steuereinheit benötigt, die alle Jalousien verwaltet. Auf diese sollte der Benutzer zugreifen können, um die manuelle Steuerung zu übernehmen. Zudem sollten hier alle Befehle an die Jalousien zentral gesendet werden, die damit eine eigene Ansteuerung brauchen. Um auf Umwelteinflüsse, wie Wind und Helligkeitswert zu reagieren. Dafür wird ein Sensor-Modul gebraucht, um die Steuereinheit auch dezentral platzieren zu können. Das gesamte System sollte möglichst einfach installierbar und anpassbar sein, sowie durch Funkbetrieb Kabelleitungen zwischen den Untersystemen vermeiden und eine einfache Einrichtung gewährleisten.

% Welche Lösung wurde gewählt?
Um ein eben beschriebenes System zu gewährleisten, wurde das System in drei Untersysteme gegliedert. Die Zentralsteuerung wurde mit einem Raspberry Pi 3 realisiert, der mit Raspbian Stretch aufgesetzt wurde. Für das Sensormodul, welches die Sensordaten erfasst und versendet, sowie für das Aktormodul für die Ansteuerung der Jalousie wurde ein ESP32 als Mikrocontroller gewählt. Zusätzlich dazu wird ein Router mit W-LAN Zugang benötigt, mit Hilfe dessen die Systeme untereinander kommunizieren können.
\chapter{Grundlagen}
\label{cha:Grundlagen}

\section{MQTT (Message Queue Telemetry Transport)}

\subsection{Allgemein}
Bei MQTT handelt es sich um ein einfaches TCP/IP basiertes Nachrichtenprotokoll, das auf den Bereich Mobile und Internet of Things zugeschnitten ist. MQTT ist gerade im Bereich der Homeautomation sehr beliebt.

Ziele von MQTT sind neben einer hohen Zuverlässigkeit ein sparsamer Umgang mit Bandbreite und Ressourcen. Dadurch eignet sich MQTT auch zur Kommunikation mit mobilen Geräten bei denen Bandbreite und Ressourcen knapp sind. Für besonders stromsparende Geräte wie beispielsweise Sensoren gibt es weiterhin das Nachrichtenprotokoll MQTT-SN (MQTT for Sensor Networks), welches auf UDP und dadurch nicht den TCP/IP Stack benötigt. Auf MQTT-SN wird näher eingegangen.

\subsection{Publish/Subscribe Verfahren}
Der Nachrichtenaustausch bei MQTT funktioniert über das Publish/Subscribe Muster. Bei Publish/Subscribe kommuniziert der Sender der Nachricht nicht direkt mit dem Empfänger, sondern wendet sich an einen Broker, welcher die Zustellung der Nachricht übernimmt. Der Broker entkoppelt so den Sender vom Empfänger der Nachrichten.

\subsection{MQTTS (Message Queue Telemetry Transport Secure)}
Bei MQTTS wird die Kommunikation mit TLS/SSL verschlüsselt. Damit wird ein abhören der Nachrichten erschwert.

\section{TLS(Transport Layer Security)}
TLS (Vorgänger SSL[Secure Sockets Layer]) ist ein Verschlüsselungsprotokoll zur Verschlüsselung einer Kommunikation. Dabei werden am Anfang Zertifikate zwischen zwei Kommunikationspartnern ausgetauscht mittels eines public key. Dabei identifizieren sich beide nacheinander, bevor dann eine verschlüsselte Verbindung aufgebaut ist.

\section{ESP32}
Der ESP32 wurde von Espressif hergestellt und besitzt ein schon eingebautes W-LAN Modul. Damit kann er entweder ein eigenes W-LAN Netzwerk aufbauen, oder mit einem vorhandenen Netzwerk kommunizieren, ohne zusätzliche Shields zu benötigen, wie ein Arduino. Der Mikrocontroller kann mit entsprechenden Librarys in der Arduino IDE programmiert werden, oder in der ESP IDF Umgebung\footnote{siehe Kapitel~\ref{cha:Installation_IDF}} mit der Sprache C.

\chapter{Anforderungen}
\label{cha:Anforderungen}

% Tabelle Requirements
% UseCase(s)

\section{Requirements}
Am Anfang mussten die Anforderungen an das System entworfen werden, um später genau zu wissen, was das fertige Produkt am Ende für Funktionen besitzen soll. Diese Anforderungen wurden in Requirements gefasst und in der Tabelle~\ref{tab:requirements} festgehalten.


% Please add the following required packages to your document preamble:
% \usepackage{booktabs}
% \usepackage[table,xcdraw]{xcolor}
% If you use beamer only pass "xcolor=table" option, i.e. \documentclass[xcolor=table]{beamer}
% \usepackage{longtable}
% Note: It may be necessary to compile the document several times to get a multi-page table to line up properly
\begin{longtable}[ht]{p{0.03\textwidth}  p{0.8\textwidth} p{0.04\textwidth} p{0.04\textwidth}}
	\captionabove[Requirements des Systems]{Requirements des Systems}\\
	\label{tab:requirements}\\
	\textit{Kategorie:}~\textbf{F}(Funktional);~\textbf{IF}(Interface);~
	\textbf{Q}(Quality)\\
	\textit{Verifikationsmethode:}~\textbf{S}(Similaris);~\textbf{I}(Inspektion);~
	\textbf{R}(Review);~\textbf{M}(Measurement);~
	\textbf{A}(Analysis);~\textbf{T}(Test)\\ 
	\\
	\toprule
	\rowcolor[HTML]{FFFC9E} 
	{\color[HTML]{333333} \textbf{Nr.}} & {\color[HTML]{333333} \textbf{Beschreibung}}  & {\color[HTML]{333333} \textbf{Kat.}} & {\color[HTML]{333333} \textbf{VM}} \\* \midrule
	\endfirsthead
	%
	\multicolumn{4}{c}%
	{{\bfseries Table \thetable\ continued from previous page}} \\
	\toprule
	\rowcolor[HTML]{FFFC9E} 
	{\color[HTML]{333333} \textbf{Nr.}} & {\color[HTML]{333333} \textbf{Beschreibung}}  & {\color[HTML]{333333} \textbf{Kat.}} & {\color[HTML]{333333} \textbf{VM}} \\* \midrule
	\endhead
	%
	1 & Das System soll modular in einen Regler, ein Sensormodul und eine Aktoransteuerung aufgeteilt sein. & F & T \\* \midrule
	
	2 & Das System soll die Position einer Jalousie zwischen 0\% und 100\% einstellen können. & F & T \\* \midrule
	
	3 & Der Sollwert für die Jalousieposition soll mit Hilfe einer Schnittstelle eine manuelle Einstellung der Jalousieposition bieten. & IF & T \\* \midrule
	
	4 & Der Sollwert für die Jalousieposition soll entsprechend der Helligkeit einstellbar sein. & IF& T \\* \midrule
	
	5 & Der Sollwert für die Jalousieposition soll entsprechend der Windstärke einstellbar sein. & IF& T \\* \midrule
	
	6 & Die aktuellen Sensorwerte sollen vom Sensormodul gemessen werden. & IF& T \\* \midrule
	
	7 & Das System soll aktuellen Sensorwerte ausgeben können. & IF & T \\* \midrule
	
	8 & Das System soll die neue Jalousieposition ausgeben können. & IF & T \\* \midrule
	
	9 & Das Sensormodul soll unabhängig vom Stromnetz betrieben werden können. & F & I \\* \midrule
	
	10 & Die Batterie des Sensormoduls soll mindestens 1 Jahr überdauern können. & F & T \\* \midrule
	
	11 & Die Batterie des Sensormoduls soll austauschbar sein. & F & I \\* \midrule
	
	12 & Das System soll 1 mal pro Sekunde aktualisiert werden. & F & T \\* \midrule
	
	13 & Das System (bis auf das Sensormodul) soll über eine externe Spannungsversorgung betrieben werden. & F & T \\* \bottomrule
		
\end{longtable}

\section{UseCase}
Die Abbildung~\ref{fig:UseCase} zeigt den normalen Anwendungsfall des Systems. Der Benutzer kann Einfluss auf die Jalousie Software nehmen, zum Beispiel über eine Smarthome Plattform oder eine Sprachsteuerung. Von dieser aus gehen Befehle über die Schnittstelle an den Regler und es werden aktuelle Informationen über den Zustand wiederum über die Schnittstelle gesendet, um sie dann auf einer Statusanzeige anzeigen zu können. Der Regler wiederum entscheidet, ob eine Handlung nötig wird und kann dementsprechend Befehle an den Aktor senden, der den Motor der Jalousie steuert. Außer dem Benutzer kann zusätzlich eine Wartungsperson direkt auf die Jalousie Steuerung zugreifen und Änderungen vornehmen.
\begin{figure}[hbt]
	\centering
	\includegraphics[width=1\linewidth]{images/UseCase}
	\caption[UseCase Diagramm]{UseCase Diagramm für den normalen Anwendungsfall.}
	\label{fig:UseCase}
\end{figure}
\chapter{Planung des Projekts}
\label{cha:Planung}

\section{Arbeitspakete}

\section{Zeitplan}

\section{Kosten}
\chapter{Architektur}
\label{cha:Architektur}

\section{Statische Architektur}
% mit Component diagramm

\section{Verhalten}
% mit state or Activity Diagram


Die Abbildung \ref{fig:Sequence_UseCase} zeigt die Kommunikation zwischen den einzelnen Modulen mithilfe von MQTTS. Die Darstellung der Kommunikation ist zur besseren Übersicht hierbei auf das minimum reduziert. Das Regler Modul dient zur Verwaltung aller Steuerungsinputs. Es kann sich hierbei sowohl um Inputs von Sensor Modulen, als auch um eine manuelle Ansteuerung handeln, die beispielsweise über eine Sprachsteuerung erfolgt. Auf Basis dieses Inputs entscheidet es welche Position die Jalousie anfahren soll. Die Anzahl an Sensor Modulen und Modulen die eine manuelle Ansteuerung vornehmen ist hierbei nicht begrenzt. Die anzusteuernde Position wird von dem Regler Modul wieder auf das entsprechende topic (message identifier) gesendet und kann anschließend von einer beliebigen Anzahl an Aktor Modulen ausgelesen werden.

Da der ESP32 über zwei Kerne verfügt und zudem bereits mit FreeRTOS ausgeliefert wird und dadurch über einen Scheduler verfügt, bietet sich die Nutzung von Tasks an. Zur Darstellung der Funktionsweise werden daher auch hier Sequenzdiagramme genutzt, die die Kommunikation zwischen den einzelnen Tasks zeigt. Abbildung \ref{fig:Sequence_UserInput} zeigt die Ansteuerung des Aktormoduls. Bei dem User handelt es sich in diesem Fall um das Regler Modul, welches die neue Position mithilfe von MQTTS verschlüsselt überträgt. Der MQTT Task liest die neue Position aus, prüft ob diese zwischen 0 und 100 Prozent liegt und überschreibt die alte Position in einer Queue der Größe 1. Der Motor Control Task liest die neue Position aus der Queue aus, steuert den Motor der Jalousie an um die entsprechende Position zu erreichen und speichert die neue Position anschließend im Flash des ESP32 um diese auch nach einem Neustart wieder abrufen zu können. Dennoch kann nicht ausgeschlossen werden, dass aufgrund eines Fehlers die aktuelle Position nicht mit der gespeicherten Position übereinstimmt. Dies hat zur Folge, dass die Jalousie außerhalb des vorgesehen Bereichs von 0-100 Prozent angesteuert wird und dadurch Schaden nimmt. Um dies zu verhindern macht es Sinn an den Bereichsenden, also bei 0 Prozent und 100 Prozent einen Endstopp einzusetzen.
Abbildung \ref{fig:SequenceMotorControl} zeigt die Implementierung der Endstopps. Der Interrupt Task wird ausgeführt wenn ein Endstopp erreicht wird. Dieser schreibt die Information welcher Interrupt getriggert wurde in eine Queue der Größe 10, so dass auch mehrere Interrupts abgespeichert und dann sequentiell abgearbeitet werden können. Der GPIO Task ist ein Task der die Interrupts aus der Queue ausliest. Handelt es sich um einen Interrupt von einem Endstopp löscht er den Motor Control Task um so zu verhindern, dass dieser den Motor weiter außerhalb des vorgesehenen Bereichs ansteuert. Anschließend korrigiert er die aktuelle Position der Jalousie und speichert diese wieder im Flash ab. Nach der Korrektur erstellt er den Motor Control Task neu, so dass dieser die Position wieder entsprechend des Inputs ansteuern kann. Ein weitere Anforderung ist es auch beim Endkunden noch auf sicherem Weg Updates auf das Endgerät einspielen zu können. Abbildung \ref{fig:Sequence_Update} zeigt wie ein entsprechendes Update erfolgt. Hierbei werden zunächst Updateinformation von einem Webserver geladen. Diese beinhalten die Information welche Version der Anwendung am Aktuellsten ist und auf welchem Server diese sich befinden. Ist die Version der aktuellsten Version neuer als die aktuell eingesetzte Version, wird diese heruntergeladen. Der ESP32 startet anschließend mit der neuen Version neu. Scheitert dies fällt er zurück auf die alte Softwareversion. Zur Absicherung der Verbindung wird hierbei tls Verschlüsselung eingesetzt. Weiterhin ist es nicht möglich einen falschen Updateserver anzugeben, da dieser lokal auf dem Gerät hinterlegt ist und durch Secure Boot in Verbindung mit Flash Encryption kann auch ein lokales Einspielen anderer Software, sowie das Auslesen von Informationen wie WLAN Passwörtern wirkungsvoll verhindert werden. Damit bleibt nur die Möglichkeit die Updateinformationen direkt auf dem Server zu manipulieren. Standardmäßig werden OTA Updates mit MD5 gehasht, was nicht mehr als sicher gilt. Für eine Endanwendung wäre es daher sinnvoll stattdessen besipielsweise sha256 zu nutzen. Die vorliegende Implementierung macht hiervon aktuell jedoch keinen Gebrauch. Weiterhin ist die Sicherheit durch die genannten Sicherheitsvorkehrungen bedeutend höher als bei den aktuell üblichen IOT Geräten.

\begin{figure}[hbt]
	\centering
	\includegraphics[width=1\linewidth]{images/Sequence_UseCase}
	\caption[Sequence UseCase]{Sequenz Diagramm für die Kommunikation zwischen den einzelnen Modulen}
	\label{fig:Sequence_UseCase}
\end{figure}

\begin{figure}[hbt]
	\centering
	\includegraphics[width=1\linewidth]{images/Sequence_UserInput}
	\caption[Sequence UserInput]{Sequenz Diagramm für die Ansteuerung des Aktormoduls}
	\label{fig:Sequence_UserInput}
\end{figure}

\begin{figure}[hbt]
	\centering
	\includegraphics[width=1\linewidth]{images/Sequence_MotorControl}
	\caption[Sequence Diagramm MotorControl]{Sequenz Diagramm der Motorsteuerung}
	\label{fig:SequenceMotorControl}
\end{figure}

\begin{figure}[hbt]
	\centering
	\includegraphics[width=0.7\linewidth]{images/Sequence_Update}
	\caption[Sequence Update]{Sequenz Diagramm zum Ablauf eines Updates}
	\label{fig:Sequence_Update}
\end{figure}

\chapter{Implementierung}
\label{cha:Implementierung}

\section{Interessante Code-Segmente}

\section{Gegenüberstellung Komplexität (SLOC)}
% 1. Schätzung, 2.Schätzung, Lösung
\chapter{Test}
\label{cha:Test}

\section{Test Procedure/Report}
% Ggf. Deployment Diagram
\chapter{Installation}
\label{cha:Installation}

\section{Deployment Diagram}
\chapter{Benutzerhandbuch}
\label{cha:Benutzerhandbuch}
% Was auch immer hier hin kommt...
% Auf jeden Fall ein Verweis auf das Github Wiki :D
\chapter{Fazit und Ausblick}
\label{cha:Fazit und Ausblick}

Der Aktuelle Stand ist zwar grundsätzlich betriebsbereit, wobei gegebenenfalls einige Anpassungen an die individuellen Gegebenheiten nötig ist. Dies betrifft sowohl die Helligkeitsmessung, da hier aktuell noch nicht getestet ist, ob hier die Helligkeitsstärken tatsächlich sinnvoll sind, als auch die Ansteuerung der Jalousie, da hier aktuell von einem bestimmten Schrittmotortreiber ausgegangen wird, wobei sich dies von Jalousie zu Jalousie unterscheiden kann. Das Regler Module besitzt bereits die Funktion getrennt mehrere Räume anzusteuern, wobei sowohl das Sensor Module, als auch das Actuator Module hiervon aktuell noch keinen Gebrauch machen. Insbesondere die Sicherheitsvorkehrungen die leider bei vielen IoT Projekten gänzlich vernachlässigt werden, sind bei dieser Realisierung jedoch bereits sehr gut, wobei leider aktuell noch ein eigener "Update Server" nötig ist auf den die binaries für OTA Updates abgelegt werden. Durch den modularen Aufbau lässt sich ein Großteil des Projektes auch auf andere IOT Geräte anwenden (Insbesondere die MQTTS Funktionalität und die OTA Updates).

Aus den oben genannten Gründen soll sich das Projekt in Zukunft auch mehr auf die Aspekte der sicheren Kommunikation und Updatefähigkeit von IOT Geräte konzentrieren um diese in anderen Projekten nur noch als Teilkomponenten einfügen zu müssen. Konkret sind hier für die nähere Zukunft folgende Erweiterung geplant:

\begin{itemize}
	\item Da der Code sowieso öffentlich auf Github zur Verfügung gestellt wird soll das Deployment mithilfe einer Continous Integration (CI) automatisiert werden, so dass die aktuellste binary als release zur Verfügung gestellt wird, wobei diese dann auch direkt für OTA Updates genutzt werden können.
	\item Die aktuellen Tests sollen automatisiert werden, so dass diese durch eine CI beispielsweise vor Updates des aktuellen Produktionscode möglich sind
	\item Aktuell muss der Nutzer noch selbst einen MQTT Broker aufsetzten mit TLS und username/password Authentification. Weiterhin muss er selbst entsprechende Skripte wie das Regler Module in diesem Projekt starten. Dieser Prozess soll in Zukunft automatisiert werden um so auch Personen ohne das nötige technische Hintergrundwissen eine einfache Installation zu ermöglichen
	\item Die Unterstützung von mehreren Räumen soll auf den MQTT Task ausgeweitet werden um so Projekten direkt eine einfache Möglichkeit zur Ansteuerung von Geräten in verschiedenen Räumen zu geben
\end{itemize}

Weitere Erweiterungen sind in vielfältiger Art und Weise möglich, wobei die eben genannten aktuell die höchste Priorität besitzen.

% ---- Literaturverzeichnis ----------

% \bibliography{literature/literatur1,literature/literatur2}
										% Einbindung mehrerer Verzeichnisse in einem \bibliography Befehl mit Kommata trennen - keine Leerzeichen nach den Kommata!
\bibliographystyle{alphadin}           	%plain: alphabetisch, unsrt: nach Zitat, alphadin: NameJahr


% -----Ausgabe aller Verzeichnisse ---
\setlength{\parskip}{0.5\baselineskip}
%\renewcommand{\indexname}{Sachwortverzeichnis}
%\printindex								% Erzeugen des Indexverzeichnises
%\addcontentsline{toc}{chapter}{\indexname}
% Version 0.1 vom 31.10.14 - T. Kibler

% Änderungswünsche bitte an T. Kibler melden

% alle Abkürzungen, die in der Arbeit verwendet werden. Nachfolgend sind allgemeine Abkürzungen aufgelistet. Die für die Arbeit spezifischen Abkürzungen sollten innerhalb des Haupttextes aufgeführt werden. Die Alphabetische Sortierung übernimmt Latex.

\nomenclature{Abb.}{Abbildung}
\nomenclature{bzw.}{beziehungsweise}
\nomenclature{DHBW}{Duale Hochschule Baden-Württemberg}
\nomenclature{ebd.}{ebenda}
\nomenclature{et al.}{at alii}
\nomenclature{etc.}{et cetera}
\nomenclature{evtl.}{eventuell}
\nomenclature{f.}{folgende Seite}
\nomenclature{ff.}{fortfolgende Seiten}
\nomenclature{ggf.}{gegebenenfalls}
\nomenclature{Hrsg.}{Herausgeber}
\nomenclature{Tab.}{Tabelle}
\nomenclature{u. a.}{unter anderem}
\nomenclature{usw.}{und so weiter}
\nomenclature{vgl.}{vergleiche}
\nomenclature{z. B.}{zum Beispiel}
\nomenclature{z. T.}{zum Teil}				% Datei mit allgemeinen Abkürzungen laden
\renewcommand{\nomname}{Verzeichnis verwendeter Formelzeichen und Abkürzungen}
\setlength{\nomlabelwidth}{.20\hsize}
\renewcommand{\nomlabel}[1]{#1 \dotfill}
\setlength{\nomitemsep}{-\parsep}
\printnomenclature						% Erzeugen des Abkürzungsverzeichnises, siehe auch Inhalt der Datei pages/abkuerzungen.tex
\cleardoublepage
%\renewcommand{\glossaryname}{Glossar}
%\printglossaries
%\cleardoublepage
\listoffigures 							% Erzeugen des Abbildungsverzeichnisses 
\cleardoublepage
\listoftables 							% Erzeugen des Tabellenverzeichnisses
\cleardoublepage

% -----Anhang ------------------------

\begin{appendix}
\clearpage
%\pagenumbering{Roman}					% große, römische Seitenzahlen für Anhang, falls gewünscht
\addchap{Anhang A}
\setcounter{chapter}{1}
\setcounter{section}{0}
\setcounter{table}{0}
\setcounter{figure}{0}

\section{Was alles rein muss}

\begin{itemize}
	\item Implementierung zum Beispiel
\end{itemize}
%\include{chapter/vorlagen/anhang_vorlagen}		% Zeile auskommentieren bei finalem Dokument!
\end{appendix}


\end{document}